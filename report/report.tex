\documentclass[a4paper,12pt]{article}

\usepackage[utf8x]{inputenc}
\usepackage[T1]{fontenc}
\usepackage{lmodern}

\usepackage{amsmath}
\usepackage{amssymb} % math symbols
\usepackage{geometry}
\usepackage{a4wide}
\usepackage{enumerate}
\usepackage{xcolor}
\usepackage{listings}
\usepackage{graphicx}
\usepackage{lastpage}

\usepackage{cite}
\usepackage{hyperref}

\usepackage{fancyhdr}
\pagestyle{fancy}
\fancyhead{}
\fancyfoot{}

\renewcommand{\contentsname}{Summary}

\lhead{Axel Angel \& Luca La Spada \& Guillaume Martres}
\rhead{Big data}
\rfoot{Page \thepage\ of \pageref{LastPage}}
\lfoot{\today}

\begin{document}
%\begin{titlepage}
\begin{center}
\sffamily


\null\vspace{2cm}
{\Huge CodecWatch \\[12pt] A multi-encoder comparator ?} \\[24pt]
\textcolor{gray}{\small{Big Data \\ École Polytechnique Fédérale de Lausanne}}

\hbox{\hspace{-20ex}
\includegraphics[width=19.5cm]{figures/White}}

\vfill

\begin{tabular} {cc}
\parbox{0.3\textwidth}{\includegraphics[width=4cm]{figures/epfl}}
&
\parbox{0.7\textwidth}{%
	by \\ [4pt]
	\hspace{3em} Martres Guillaume (203085)\\
	\hspace{3em} Angel Axel (201284)\\
	\hspace{3em} La Spada Luca (193276)\\[9pt]
\small

Lausanne, EPFL, \today}
\end{tabular}
\end{center}
\vspace{2cm}
\end{titlepage}

\tableofcontents

\newpage

\section{Introduction}
% Project presentation and goals
New and emerging video codec standards like
HEVC/H.265~ \cite{hevc} (MPEG standard),
VP9~ \cite{vp9} (open, royalty-free codec
developed by Google) or Daala~ \cite{daala} (research
project developed in the open by Mozilla) have caused a great increase of
activity around open source video encoders for these formats. The main challenge
of any encoder is to provide the greatest visual quality at acceptable file size
and encoding speed. This means that all these projects must measure their output
quality and compare it with the state of the art. It would be a great boost to
productivity for developers if an open source project to compare and track the
quality of these evolving encoders existed. This is the goal of CodecWatch,
which provides the following functionalities:
\begin{itemize}
\item Do nightly encode of a standard set of video clips using the latest git
revision of various open source encoders, at various bitrates.
\item Display in graph forms various metrics (like
    PSNR~ \cite{psnr} or
    SSIM~ \cite{ssim})
\item As quality metrics are only an approximation of the actual visual quality,
provide a way to compare the encoded videos side-by-side.
\end{itemize} Instead of starting from scratch, we implemented CodecWatch by
extending the existing open source encoding platform
OSCIED~ \cite{oscied}.

\section{OSCIED description}
% Why we choose oscied, architecture summary
The goal of OSCIED is to provide a distributed cloud-powered platform to encode,
publish and manage distributed resources.  In practice it means that medias are
first uploaded to the platform, then multiple encoding jobs can be dispatched to
transformer machines and done in parallel.  The resulting medias are then
available in the platform for further processing or direct download by the
administrator.  Moreover these files can be published on any publisher machine
for public access.
% TODO: develop more
For more details, we invite the reader to consult the original paper by David
Fischer~\cite{df_thesis}.

% Explain we had to understand how OSCIED works, architecture complexity, it's enormous

\section{Deployment}
\subsection{OSCIED}
% what OSCIED provides to deploy, locally works fine
As the goal of OSCIED is to distribute computer resources across multiple
machines, the platform provides high-level scripts to deploy the different roles
locally or to a cloud provider.  Behind the scene OSCIED uses juju, a
provisioning and deployment tool that supports major cloud providers.  We won't
describe in detail how juju works but basically it is made of two parts: the
internals capable of interfacing with the cloud APIs and lots of
role-specific files (charms).  The charms are in fact description files which
tells the system how to deploy the different roles: package dependencies and how
to startup the services.  For example the webui of OSCIED needs a web server
thus the webui charm will tell juju to install Apache with PHP and setup the web
files in the right directory.

We first began to deploy a new instance of OSCIED with all roles locally on our
server at EPFL, each role inside its own LXC (linux container).  This involved
numerous bug fixes to OSCIED which where then merged upstream, see
\cite{ebu_merge1},
\cite{oscied_merge1}
and
\cite{pytoolbox_merge1}.

\subsection{Use of Microsoft Azure}
As video encoding is a very CPU intensive task, we had planned to use our
provided Azure instances to increase the number of videos and encoders we could
demonstrate our project on. Azure support required us to use a recent version of
Ubuntu and thus update OSCIED to account for changes in its dependencies (like
the removal of the -sameq argument from ffmpeg). This was completed
successfully, but unfortunately, we did not manage to properly deploy OSCIED on
Azure on time. This is due mostly to the quite complex and fragile scripts that
are normally used to deploy OSCIED and bugs in juju itself, like
\cite{juju_bug1} which we reported ourselves.

%\section{Adaptations}
% modification we made, how we changed

\subsection{Functional modifications of OSCIED}
% TODO: Axel
To integrate our goals into OSCIED we decided to develop directly into the live
containers, we had direct feedback of our modifications.  Multiple roles were
affected: Orchestra, Transformer and Webui.  We needed to collect statistics
such as the video bitrate, quality measures such as PSNR/SSIM, media source such
as the git repository and commit, directly into the stored metadata of the media
at encoding time (these are pushed into the MongDB of the Orchestra role after
completion)

We first had to understand the internal implementation: hierarchy of files, the
classes and how they all work together.  This took us some time and finally we
were able to modify exactly what we wanted.  There were multiple obstacles due
to the dynamic nature of Python such as multiple level of indirections of the
API (callbacks) and the lack of type description, these components are lazily
coupled and they break only at runtime after a job.  Thus we had to debug our
code in live by looking at the logged files and iterate until we had what we
wanted.

First we decided to modify the Transformer code where the encoding process is
triggered after which the metadata are computed.  It is written in Python and
backed up by Celery, thus the call are all asynchronous and a callback is
called.  We modified the encoding target (ffmpeg) to collect more metadata
(called measures), these are computed by specialized programs of the Daala
project.

Secondly we had to modify Orchestra to accept these measures and put it as
regular metadata into the database. Some part of the code base is shared with
the Transformer and thus are written in Python but backed up by Flask (a light
server designed for serving APIs).

Thirdly the web interface Webui had to modified to display these informations,
this part is written in PHP.  Columns were added in the media panel of the
interface in order to show the results of our tests and for the final users
so that they can compare numeric values directly.  As textual information is not
sufficient, we wrote a small API that serves these measures directly to a
separate interface with graphs.  We had to integrate our work with the PHP
framework of the Webui (called CodeIgniter), that is: register new routes and
ensure we provide a json output correctly to interpolate with Javascript (see
the next section).

\subsection{Graph representation} As describe previously, the statistics are
only viewable though a big table. Hence, we create our own graph to display
these data which simplify greatly how encoders behave. The graph is a plot that
shows you the evolution of the quality of different encoder on some specific
video called sample. These results are retrieved from the OSCIED database. The
graph gives you the possibility to filter the result by sample, by encoder, by
metric. Moreover, you can set interval to retrieve only encoding that have been
done in a specific range of dates.

The graph is coded principally in HTML, Javascript and CSS. It uses different
Javascript plugins: Chosen~\cite{harvesthq_chosen}
that change the default style of a multiselect in something with more hype,
jQueryUI~\cite{jqueryui},
TimePicker~\cite{timepicker} that
improves the default datepicker of jQueryUI in a more detailed one (you can set
even the second), jQuery~\cite{jquery} and
flotcharts~\cite{flotchart} that is a jQuery plugin
used to create plots. Futhermore, we take areweslimyet.com~\cite{areweslimyet} as inspiration.

In figure~\ref{fig:graph1}, you can see an example of such graph. The graph
displays in the X-axis the bitrate and in the Y-axis the decibel of videos. If
you pass your mouse over a point a tooltip appears with the exact values of the
point.

\begin{figure}[!h] \centering
  \includegraphics[width=1\textwidth]{figures/graph1.png}
  \caption{Example of graph rendering}
  \label{fig:graph1}
\end{figure}

\subsection{SplitView} Doing a graph is a good idea to compare encoder, but actually these are only number, that why we adapted the well-know visual comparator SplitView~\cite{splitview} to our project. The principles of SplitView is quite simple, as you can see in figure~\ref{fig:split1}, you have a slider that cut literrealty the video in two parts. Each part rendered a specific video that we specify in the comboboxes above them. You can move the slide to change the ratio.

Furthermore, we add a feature to SplitView. We have called it \emph{The Blind Test}. The blind test is all about the green button that we can see in figure~\ref{fig:split1}. When we click it, the green square around that contains the information about the videos becomes invisible and the two videos are swapped with probability $0.5$. Thanks to this feature you loose completely the notion of which would be the best one based on his intrinsic value. When you reclick on the blind test button, the solution reappears.

\begin{figure}[!h] \centering
  \includegraphics[width=1\textwidth]{figures/split1.png}
  \caption{SplitView}
  \label{fig:split1}
\end{figure}

\section{Future work}
Beside support for Windows Azure, there's a few possible improvements we could
have done if we had had enough time and resources:
\subsection{Chunked encoding}
The most efficient way to parallelize video encoding is to split the source into
chunks which each last a few seconds, encode these chunks separetly, then merge
them. This does not negatively affect the video quality since video streaming
requires that segments of more than a few seconds are independently decodable
anyway (otherwise seeking to a particular point in a video would be too slow,
since you would have to retrieve and decode everything before that point in the
video first). CodecWatch does not implement this, although this wouldn't be too
hard to add (ffmpeg already supports chunking a video, and tools like mkvtoolnix
or l-smash can be used to merge it back together). This is not a critical
problem because CodecWatch is mostly designed for benchmarking purposes and this
is usually done by using very short clips~\cite{derf}.
\subsection{Easier configuration of the nightly encodes}
Right now, it's easy for an user to change the set of video that is used for
benchmarking by uploading or deleting videos in the web interface, but there is
no interface to change the encoding parameters used (quality, bitrate, ...),
instead the cron\_enco.py script has to be modified manually.

\section{Conclusion}
% what we made, sumup, how can be used and continued, improved
% what we did is useful, blabla
We managed to produce a useful proof of concept for what a benchmarking platform
for encoders should look like. While the use of OSCIED allowed us to avoid
having to design our own distributed encoding system, it turned out to be harder
to adapt to our needs that anticipated. Thus, it might make sense for future
work to rebuild the server side from scratch using modern deployment tools like
CoreOS~\cite{coreos} from scratch but keep the client side (the metrics graphs and SplitView).
\appendix
\bibliography{report}{}
\bibliographystyle{plain}

\end{document}
